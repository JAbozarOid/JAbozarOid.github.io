\documentclass[]{article}
\usepackage{lmodern}
\usepackage{amssymb,amsmath}
\usepackage{ifxetex,ifluatex}
\usepackage{fixltx2e} % provides \textsubscript
\ifnum 0\ifxetex 1\fi\ifluatex 1\fi=0 % if pdftex
  \usepackage[T1]{fontenc}
  \usepackage[utf8]{inputenc}
\else % if luatex or xelatex
  \ifxetex
    \usepackage{mathspec}
  \else
    \usepackage{fontspec}
  \fi
  \defaultfontfeatures{Ligatures=TeX,Scale=MatchLowercase}
\fi
% use upquote if available, for straight quotes in verbatim environments
\IfFileExists{upquote.sty}{\usepackage{upquote}}{}
% use microtype if available
\IfFileExists{microtype.sty}{%
\usepackage{microtype}
\UseMicrotypeSet[protrusion]{basicmath} % disable protrusion for tt fonts
}{}
\usepackage[unicode=true]{hyperref}
\hypersetup{
            pdfborder={0 0 0},
            breaklinks=true}
\urlstyle{same}  % don't use monospace font for urls
\usepackage{graphicx,grffile}
\makeatletter
\def\maxwidth{\ifdim\Gin@nat@width>\linewidth\linewidth\else\Gin@nat@width\fi}
\def\maxheight{\ifdim\Gin@nat@height>\textheight\textheight\else\Gin@nat@height\fi}
\makeatother
% Scale images if necessary, so that they will not overflow the page
% margins by default, and it is still possible to overwrite the defaults
% using explicit options in \includegraphics[width, height, ...]{}
\setkeys{Gin}{width=\maxwidth,height=\maxheight,keepaspectratio}
\IfFileExists{parskip.sty}{%
\usepackage{parskip}
}{% else
\setlength{\parindent}{0pt}
\setlength{\parskip}{6pt plus 2pt minus 1pt}
}
\setlength{\emergencystretch}{3em}  % prevent overfull lines
\providecommand{\tightlist}{%
  \setlength{\itemsep}{0pt}\setlength{\parskip}{0pt}}
\setcounter{secnumdepth}{0}
% Redefines (sub)paragraphs to behave more like sections
\ifx\paragraph\undefined\else
\let\oldparagraph\paragraph
\renewcommand{\paragraph}[1]{\oldparagraph{#1}\mbox{}}
\fi
\ifx\subparagraph\undefined\else
\let\oldsubparagraph\subparagraph
\renewcommand{\subparagraph}[1]{\oldsubparagraph{#1}\mbox{}}
\fi

\date{}

\begin{document}

\textbf{From page 67 to 69}

E-researchers who obtain consent over the Net also have to be aware of
the risk of possibly attracting vulnerable populations to their study.
As Roberts (2000a) points out, important demographic details, such as
age, may be concealed by potential participants. This may lead to
vulnerable populations (e.g., children or persons of diminished mental
capacity) being recruited and included in a study without the
researcher's knowledge. Schrum (1995) maintans that this alone presents
serious problems in obtaining some degree of informed consent and
considers this to be the most difficult ethical issue of online
research. While the e-researcher will need to acknowledge this as a
looming possibility (or a limitation) to conducting Net-based research,
Roberts (2000b) maintains there is a counter argument. While attracting
vulnerable participants is an ever-present possibility, the Internet
also has the ability to access participants who might otherwise be
unable to participate or who traditionally may not have been able to
have a voice in research projects. For a variety of reasons(e.g.,
geographic, disabilities, situational) researchers are sometimes not
able to access specific people or populations. In certain circumstances,
Net-base research can provide greater inclusivity by accessing these
populations.

Some researchers who wish to obtain consent have creatively used forms
feature on Web pages to obtain information. Figure 5.1 is an example of
an online consent form by Nora Boekhout
(\url{http://www.teacherwebsheIf.com/}) at Simon Fraser University (for
consent form see
\url{http://modena.intergate.ca/personal/boekhout/technologhyincurruculum/ethicsforms.html}).
While the information on the form is standard for Simon Fraser
University, notice that this online consent from has a section for a
witness information. While there are no guarantees that the witness is
credible (or even exists for that matter) this does provide another
participation as contrasted with the way most of us casually click past
license information (without reading) when first using new software
packages purchased or downloaded from the Net.

\includegraphics[width=6.49951in,height=3.19403in]{media/image1.PNG}

Finally, it needs to be noted that there are certain circumstances when
the \textbar{}Ne should probably note be used to obtain consent. These
circumstances include situations where the consent of guardians of
minors or that of persons of diminished mental capacity is required. We
advise against this for the protection of both the e-researcher and the
participant.

\textbf{WHEN IS CONSENT NEEDED? THE PUBLIC VERSUS PRIVATE DILEMMA}

Informed consent is needed in almost all types of research with notable
exceptions.

First, consent is generally not required to study an activity that is
nonintrusive and takes palce in a public space. For example, it is not
necessary to obtain consent when undertaking naturalistic
observations(for example, when studying linguistic patterns of fans'
chanting at a football game). Neither it is usually necessary to obtain
permission when studying a public record or archive. For example, it is
not necessary to obtain permission to study the public speeches of
politicians, perform content analysis of newspapers or other mass media,
or to study the public record of proceedings from a legislature. It is
possible to argue that this notion of public space is appropriately
extend to the study of activity on public newsgroups, mail lists, chat
rooms, or virtual reality environments (e.g, MOOs, MUDs). Specifically,
as these kinds of online spaces are open for anyone to join and, hence,
can be interpreted as public spaces, informed consent from every
participant is not required since the researcher is often not
participating and , thus, not affecting the interaction that takes
place.

Or is it a public space? Thus interpretation is not as straightforward
as some e-researchers would like it to be, as the sense of what is
public or private is defined not by the technology, but by the
perception of privacy and inclusion that us maintained by the
participants. Imagine, for a a moment, that you are in a public park and
you need to use the public washroom. As you are leaving this public
facility you notice there is a video camera in each of the washroom
cubicles. How do you think you might feel to learn that this is a part
of a research project? As King (1996) notes, with this kind of example
``the sense of violation possible is proportional to the expectation of
privacy that group members had prior to learning they were studied''.
For example, studies with virtual self-help groups have shown remarkable
candor among participants and the publication of this content has been
viewed by some participants as a violation of the privacy of the group
(Sharf,1999). An additional factor determining private space is the
degree of intimacy that the researcher is studying. King notes that,
generally, activity in a public place does not require informed consent.
For example, noting how people are sitting on park benches. Nevertheless
as Waskul and Douglass (1996) point out, if one installs a tape recorder
and records conversations that take place on the park bench, a much
different level of consent is required for ethical research.

Waskul and Douglass (1996) remind us further that ``ethical
considerations should entail an interplay between codes of conduct and
an intimate understanding of the nature of the online environment.'' To
behave ethically requires explicit and expert knowledge of the context
within the researcher functions. The Net is made up of a diverse set of
technological and cultural contexts. For example, the ethics of
analyzing the interaction in a large public discussion board sponsored
by a media outlet such as the New York Time, call for far different
means to protect privacy than research involving private emails.
Further, codes of conduct may apply differently to different types of
research. For example, the study of anonymous language use in public
online chat rooms and the publication of results requires different
level of individual disclosure than a study that is focused on
identifying appropriate teacher/student interventions during an
industrial class using the same Net-based chat technology. Thus, even
though the technology is the same, different standards of ethical
research behavior are required for these different research
investigations.

To help the e-researcher determine when and what type of consent is
required, many of the formal professional and research granting bodies
provide guidelines that can help address some of the gray areas of
ethical research. The 1994 Canadian Code of Ethical Conduct for Research
Involving Humans
(\url{http://www.nserc.ca/programms/ethics/english/policy.htm}) defines
research participants as ``living individuals or groups of living
individuals about whom a scholar conducting research obtains(1) data
through intervention or interaction with the individual or group, or (2)
identifiable private information.'' Applying these guidelines can be
helpful in determining when naturalistic observations (as, for example,
nothing the length of posting or language used in public Usenet groups)
become personal interventions. If the researcher has no interaction,
then it is generally not necessary to obtain informed consent from the
participants. Unfortunately, ethical issues are sometimes very
complicated in Net-based research, making it unclear how to apply
existing consent guidelines. In these cases, judgment calls must be made
to defend choices that require or dispense with requirements for
consent. The American Psychology Association ethical code for
researchers (Draft 6.1 at \url{http://www.apa.org/ethics/}) notes that
``before determining that planned research (such as research involving
only anonymous questionnaires, naturalistic observations, or certain
kinds of archival research) does not require the informed consent of
research participants, psychologist consider applicable regulations and
institutional review board requirements, and they consult with
colleagues as appropriate.'' In keeping with this guideline, it is
advisable that the e-researcher consult with colleague and institutional
review boards prior to dispensing with consent.

REDUCING THE POTENTIAL TO HARM

The second core value that underlines e-research is to insure that the
e-researcher avoids, through the research process, possible harm to
research participants or non-participants who are affected by the
researcher's activities. The most common form of harm comes from
inadvertent or purposeful exposure of the participants in ways that are
perceived by those involved as damaging or hurtful. Examples of harm may
include not only physical injuries but also loss of privileges (an
inability to participate in an activity), inconvenience (i.e., wasted
time, frustration, boredom), psychological injuries (insults, loss of
self-esteem, embarrassment), economic losses (job, entrance into
programs), or legal risks (Bickman \& Rog, 1998). Further, it should not
be assumed that it is only individuals that can be harmed by such
exposure.

\end{document}
